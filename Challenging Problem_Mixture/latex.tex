\documentclass[journal,12pt,twocolumn]{IEEEtran}

\usepackage{setspace}
\usepackage{gensymb}
\singlespacing
\usepackage[cmex10]{amsmath}

\usepackage{amsthm}

\usepackage{mathrsfs}
\usepackage{txfonts}
\usepackage{stfloats}
\usepackage{bm}
\usepackage{cite}
\usepackage{cases}
\usepackage{subfig}

\usepackage{longtable}
\usepackage{multirow}

\usepackage{enumitem}
\usepackage{mathtools}
\usepackage{steinmetz}
\usepackage{tikz}
\usepackage{circuitikz}
\usepackage{verbatim}
\usepackage{tfrupee}
\usepackage[breaklinks=true]{hyperref}
\usepackage{graphicx}
\usepackage{tkz-euclide}

\usetikzlibrary{calc,math}
\usepackage{listings}
    \usepackage{color}                                            %%
    \usepackage{array}                                            %%
    \usepackage{longtable}                                        %%
    \usepackage{calc}                                             %%
    \usepackage{multirow}                                         %%
    \usepackage{hhline}                                           %%
    \usepackage{ifthen}                                           %%
    \usepackage{lscape}     
\usepackage{multicol}
\usepackage{chngcntr}

\DeclareMathOperator*{\Res}{Res}

\renewcommand\thesection{\arabic{section}}
\renewcommand\thesubsection{\thesection.\arabic{subsection}}
\renewcommand\thesubsubsection{\thesubsection.\arabic{subsubsection}}

\renewcommand\thesectiondis{\arabic{section}}
\renewcommand\thesubsectiondis{\thesectiondis.\arabic{subsection}}
\renewcommand\thesubsubsectiondis{\thesubsectiondis.\arabic{subsubsection}}


\hyphenation{op-tical net-works semi-conduc-tor}
\def\inputGnumericTable{}                                 %%

\lstset{
%language=C,
frame=single, 
breaklines=true,
columns=fullflexible
}
\begin{document}


\newtheorem{theorem}{Theorem}[section]
\newtheorem{problem}{Problem}
\newtheorem{proposition}{Proposition}[section]
\newtheorem{lemma}{Lemma}[section]
\newtheorem{corollary}[theorem]{Corollary}
\newtheorem{example}{Example}[section]
\newtheorem{definition}[problem]{Definition}

\newcommand{\BEQA}{\begin{eqnarray}}
\newcommand{\EEQA}{\end{eqnarray}}
\newcommand{\define}{\stackrel{\triangle}{=}}
\bibliographystyle{IEEEtran}
\raggedbottom
\setlength{\parindent}{0pt}
\providecommand{\mbf}{\mathbf}
\providecommand{\pr}[1]{\ensuremath{\Pr\left(#1\right)}}
\providecommand{\qfunc}[1]{\ensuremath{Q\left(#1\right)}}
\providecommand{\sbrak}[1]{\ensuremath{{}\left[#1\right]}}
\providecommand{\lsbrak}[1]{\ensuremath{{}\left[#1\right.}}
\providecommand{\rsbrak}[1]{\ensuremath{{}\left.#1\right]}}
\providecommand{\brak}[1]{\ensuremath{\left(#1\right)}}
\providecommand{\lbrak}[1]{\ensuremath{\left(#1\right.}}
\providecommand{\rbrak}[1]{\ensuremath{\left.#1\right)}}
\providecommand{\cbrak}[1]{\ensuremath{\left\{#1\right\}}}
\providecommand{\lcbrak}[1]{\ensuremath{\left\{#1\right.}}
\providecommand{\rcbrak}[1]{\ensuremath{\left.#1\right\}}}
\theoremstyle{remark}
\newtheorem{rem}{Remark}
\newcommand{\sgn}{\mathop{\mathrm{sgn}}}
\providecommand{\abs}[1]{\left\vert#1\right\vert}
\providecommand{\res}[1]{\Res\displaylimits_{#1}} 
\providecommand{\norm}[1]{\left\lVert#1\right\rVert}
%\providecommand{\norm}[1]{\lVert#1\rVert}
\providecommand{\mtx}[1]{\mathbf{#1}}
\providecommand{\mean}[1]{E\left[ #1 \right]}
\providecommand{\fourier}{\overset{\mathcal{F}}{ \rightleftharpoons}}
%\providecommand{\hilbert}{\overset{\mathcal{H}}{ \rightleftharpoons}}
\providecommand{\system}{\overset{\mathcal{H}}{ \longleftrightarrow}}
	%\newcommand{\solution}[2]{\textbf{Solution:}{#1}}
\newcommand{\solution}{\noindent \textbf{Solution: }}
\newcommand{\cosec}{\,\text{cosec}\,}
\providecommand{\dec}[2]{\ensuremath{\overset{#1}{\underset{#2}{\gtrless}}}}
\newcommand{\myvec}[1]{\ensuremath{\begin{pmatrix}#1\end{pmatrix}}}
\newcommand{\mydet}[1]{\ensuremath{\begin{vmatrix}#1\end{vmatrix}}}
\numberwithin{equation}{subsection}
\makeatletter
\@addtoreset{figure}{problem}
\makeatother
\let\StandardTheFigure\thefigure
\let\vec\mathbf
\renewcommand{\thefigure}{\theproblem}
\def\putbox#1#2#3{\makebox[0in][l]{\makebox[#1][l]{}\raisebox{\baselineskip}[0in][0in]{\raisebox{#2}[0in][0in]{#3}}}}
     \def\rightbox#1{\makebox[0in][r]{#1}}
     \def\centbox#1{\makebox[0in]{#1}}
     \def\topbox#1{\raisebox{-\baselineskip}[0in][0in]{#1}}
     \def\midbox#1{\raisebox{-0.5\baselineskip}[0in][0in]{#1}}
\vspace{3cm}
\title{AI5002: Challenging Problem: Mixture}
\author{Debolena Basak\\ AI20RESCH11003}
\maketitle
\newpage
\bigskip
\renewcommand{\thefigure}{\theenumi}
\renewcommand{\thetable}{\theenumi}
Download all Python codes from 
\begin{lstlisting}
https://github.com/Debolena/AI5002-Probability-and-Random-Variables/blob/main/Assignment_11/assignment11_code_drawing%20balls.py
\end{lstlisting}
%
and latex-tikz codes from 
%
\begin{lstlisting}
https://github.com/Debolena/AI5002-Probability-and-Random-Variables/tree/main/Challenging%20Problem_Mixture
\end{lstlisting}
\section{Problem}
Let $X \sim Bin \brak{5, \frac{1}{2}}$ and $Y \sim U\brak{0,1}$. Then $\frac{P(X+Y) \le 2}{P(X+Y) \ge 5}$ is equal to?
\section{Solution}
As $X\sim Bin\brak{5, \frac{1}{2}}$,
\begin{align}
    P\brak{X=k} = \binom{5}{k}\brak{\frac{1}{2}}^5 \text{, as n=5 and } p =\frac{1}{2}
\end{align}
$Y\sim U\brak{0,1}$. So, the CDF of $Y$ is:
\begin{align}
    F_Y \brak{y} &= P\brak{Y \le y}\\
    & = \begin{cases}
      0, & \text{if}\ y\le 0 \\
      y,  & \text{if}\ 0<y<1 \\
      1 & \text{if}\ y\ge 1
    \end{cases}
\end{align}
\begin{align}
    &P\brak{X+Y \le 2} =\sum_{k=0}^{2} P\brak{X=k, Y\le 2-k}\\
    &= \sum_{k=0}^{2} P\brak{X=k}.P\brak{Y\le 2-k}\\
    &= \binom{5}{0}\brak{\frac{1}{2}}^5. 1 + \binom{5}{1}\brak{\frac{1}{2}}^5.\brak{2-1} + \binom{5}{2}\brak{\frac{1}{2}}^5.\brak{2-2}\\
    &= \brak{\frac{1}{2}}^5 \sbrak{1+ 5\times 1 + 0 }\\
    &= \frac{6}{32}\label{eq: star}
\end{align}
Now,
\begin{align}
    P\brak{X+Y\ge 5} &= 1-P\brak{X+Y<5}\\
    &= 1-P\brak{X+Y\le4} \label{eq1}
\end{align}
\begin{align}
    P\brak{X+Y \le 4} &=\sum_{k=0}^{4} P\brak{X=k, Y\le 4-k}\\
    &=\sum_{k=0}^{4} P\brak{X=k} P\brak{Y\le 4-k}\\
    &= \sum_{k=0}^{4} \binom{5}{k} \brak{\frac{1}{2}}^5 P\brak{Y\le 4-k} \\
    &= \brak{\frac{1}{2}}^5 \sbrak{ 1\times1 + 5\times1 + 10\times 1 + 10\times1 + 0}\\
    &= \frac{26}{32} \label{eq2}
\end{align}
Using \eqref{eq1} and \eqref{eq2},
\begin{align}
    P\brak{X+Y\ge 5} &= 1 -\frac{26}{32}\\
    &= \frac{6}{32} \label{eq:double star}
\end{align}
From \eqref{eq: star} and \eqref{eq:double star},
\begin{align}
    \frac{P(X+Y) \le 2}{P(X+Y) \ge 5} = 1
\end{align}
\end{document}

